\documentclass[12pt]{article}
\usepackage{amsmath} % For equation*
\usepackage{url}     % For the web sites in the bibliography
\usepackage{verbatim}% For listing head of abyss
\usepackage[pdftex]{graphicx,color} % For including pdfs

\title{An Examination of Zipf's Law:\\ or\\ Using Gnu Make to Build a
  Document}
\author{A Learner}

\begin{document}

\maketitle

\section{Using Make}
\label{sec:make}

The point of this document is that one can build a PDF version of it
by issuing
\begin{verbatim}
$ make
\end{verbatim}
from a command line in the directory
\emph{make-novice/code/10-latex-example} which contains the
source files reported by \emph{git ls-files}
%
\input{file_list.tex}
%
The files are derived from and rely on the \emph{Software
  Carpentry}\cite{software-carpentry} lesson \emph{make-novice} which
you can obtain from \emph{git-hub}.  The following sequence of
commands will fetch and build the document:
\begin{verbatim}
$ git clone -b latex git@github.com:fraserphysics/make-novice.git
$ cd make-novice/code/10-latex-example
$ make
\end{verbatim}

The remainder of the document is an imitation research manuscript on a
study of the frequency of occurrence of words in English.

\section{Zipf's Law}
\label{sec:zipf}

Zipf's Law\cite{Zipf} concerns the frequency with which different
words occur in a language.  Here we test the law by sorting words in
texts based on the number of times that they occur.  Code from the
Software Carpentry lesson makes tables in which the first column,
\emph{w}, is a word and the second column, \emph{c}, is a count of the
number of times the word occurs.  The table is sorted so that \emph{c}
is monotonically decreasing.  Zipf observed that
\begin{equation*}
  \frac{c[n]}{c[m]} \approx \frac{m}{n}.
\end{equation*}

We use the following books in the study:
\begin{description}
\item[abyss] The People of the Abyss by Jack
  London\cite{London}
\item[isles] A Journey to the Western Islands of Scotland by Samuel
  Johnson\cite{Johnson}
\item[last] Scott's Last Expedition Volume I by Robert Falcon
  Scott\cite{Scott}
\item[sierra] My First Summer in the Sierra by John Muir\cite{Muir}
\end{description}

Here are the first ten values of \emph{w} and \emph{c} for the book
\emph{abyss}
%
\verbatiminput{abyss.head}
%
A plot of that data appears in Figure~\ref{fig:abyss}.

Counts of the two most frequent words in each book appear in
Table~\ref{tab:results}, and plots of the ten most frequent words for
the books isles, last and sierra appear in
Figures~\ref{fig:isles}-\ref{fig:sierra}.
\begin{table}
  \centering
  \begin{tabular}{c|rrr}
    \input{results.tex}
  \end{tabular}
  \caption{Counts of the two most frequent words in each book.  Zipf's
    law predicts that the ratio will be about 2.0}
  \label{tab:results}
\end{table}

\newcommand{\countfig}[1]{
\begin{figure}
  \centering
    \resizebox{0.7\columnwidth}{!}{\includegraphics{#1.pdf}}
    \caption{Counts of the ten most frequent words in the book \emph{#1}.}
    \label{fig:#1}
\end{figure}
}

\countfig{abyss}
\countfig{isles}
\countfig{last}
\countfig{sierra}

\bibliographystyle{unsrturl}
\bibliography{local}
\end{document}

%%%---------------
%%% Local Variables:
%%% eval: (TeX-PDF-mode)
%%% eval: (setq ispell-personal-dictionary "./localdict")
%%% End:
