\documentclass[]{article}
\usepackage{amsmath,amsfonts,afterpage}
\usepackage{url}
%\usepackage{showlabels}
\usepackage[pdftex]{graphicx,color} \title{An Examination of Zipf's
  Law:\\ or\\ Using Gnu Make to Build a Document}

\author{A Learner}
\begin{document}
\maketitle

\section{Using Make}
\label{sec:make}

The point of this document is that one can build a PDF version of it
by issuing
\begin{verbatim}
$ make
\end{verbatim}
from a command line in the directory
\emph{make-novice/code/09-conclusion-challenge-3} which contains the
source files reported by \emph{git ls-files} \input{file_list.tex} The
files are derived from and rely on the \emph{Software
  Carpentry}\cite{software-carpentry} lesson \emph{make-novice} which
you can obtain from \emph{git-hub}.  The following sequence of
commands will fetch and build the document:
\begin{verbatim}
$ git clone -b latex git@github.com:fraserphysics/make-novice.git
$ cd make-novice/code/09-conclusion-challenge-3
$ make
\end{verbatim}

The remainder of the document is an imitation research manuscript on a
study of the frequency of occurrence of words in English.

\section{Zipf's Law}
\label{sec:zipf}

Zipf's Law\cite{zipf:49a} concerns the frequency with which different
words occur in a language.  Here we test the law by sorting words in
texts based on the number of times that they occur.  We get arrays
\emph{w} and \emph{c} with $w_1$ being the word that occurs most
often and $c_1$ being the number of times that word occurs, $w_2$
being the second most frequent word and $c_2$ being the number of
times that word occurs, etc.  Zipf claimed that
\begin{equation*}
  \frac{c[n]}{c[m]} \approx \frac{m}{n}.
\end{equation*}

We use the following books in the study:
\begin{description}
\item[abyss] The People of the Abyss by Jack
  London\cite{London}
\item[isles] A Journey to the Western Islands of Scotland by Samuel
  Johnson\cite{Johnson}
\item[last] Scott's Last Expedition Volume I by Robert Falcon
  Scott\cite{Scott}
\item[sierra] My First Summer in the Sierra by John Muir\cite{Muir}
\end{description}

Counts of the two most frequent words in each book appear in Table
\ref{tab:results}
\begin{table}
  \centering
  \begin{tabular}{c|rrr}
    \input{results.tex}
  \end{tabular}
  \caption{Counts of the two most frequent words in each book.  Zipf's
    law predicts that the ratio will be about 2.0}
  \label{tab:results}
\end{table}

\newcommand{\countfig}[1]{
\begin{figure}
  \centering
    \resizebox{0.7\columnwidth}{!}{\includegraphics{#1.pdf}}
    \caption{Counts of the 10 most frequent word in the book \emph{#1}.}
    \label{fig:#1}
\end{figure}
}

\countfig{abyss}
\countfig{isles}
\countfig{last}
\countfig{sierra}

\bibliographystyle{unsrturl}
\bibliography{local}
\end{document}

%%%---------------
%%% Local Variables:
%%% eval: (TeX-PDF-mode)
%%% eval: (setq ispell-personal-dictionary "./localdict")
%%% End:
