\documentclass[11pt]{article}
\usepackage{amsmath,amsfonts,afterpage}
%\usepackage{showlabels}
\usepackage[pdftex]{graphicx,color}
\title{An Examination of Zipf's Law}

\author{A Learner}
\begin{document}
\maketitle

Zipf's Law concerns the frequency with which different words occur in
a language.  Here we test the law by sorting words in texts based on
the number of times that they occur.  We get arrays \emph{w} and
\emph{c} with $w[1]$ being the word that occurs most often and $c[1]$
being the number of times that word occurs, $w[2]$ being the second
most frequent word and $c[2]$ being the number of times that word
occurs, etc.  Zipf claimed that
\begin{equation*}
  \frac{c[n]}{c[m]} \approx \frac{m}{n}.
\end{equation*}

We use the following books in the study:
\begin{description}
\item[abyss] The People of the Abyss by Jack London
\item[isles] A Journey to the Western Islands of Scotland by Samuel
  Johnson
\item[last] Scott's Last Expedition Volume I by Robert Falcon Scott
\item[sierra] My First Summer in the Sierra by John Muir
\end{description}

Counts of the two most frequent words in each book appear in Table
\ref{tab:results}
\begin{table}
  \centering
  \begin{tabular}{c|rrr}
    \input{results.txt}
  \end{tabular}
  \caption{Counts of the two most frequent words in each book.  Zipf's
    law predicts that the ratio will be about 2.0}
  \label{tab:results}
\end{table}

\newcommand{\countfig}[1]{
\begin{figure}
  \centering
    \resizebox{0.7\columnwidth}{!}{\includegraphics{#1.png}}
    \caption{Counts of the 10 most frequent word in the book \emph{#1}.}
    \label{fig:#1}
\end{figure}
}

\countfig{abyss}
\countfig{isles}
\countfig{last}
\countfig{sierra}

\end{document}

%%%---------------
%%% Local Variables:
%%% eval: (TeX-PDF-mode)
%%% eval: (setq ispell-personal-dictionary "./localdict")
%%% End:
